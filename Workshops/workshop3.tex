\documentclass{article}
\usepackage[utf8]{inputenc}
% make subsections alphabetic:
\renewcommand{\thesubsection}{\thesection.\alph{subsection}}

% vector drawing package
\usepackage{tikz}
\usepackage{tikz-3dplot}    
% documentation: 
%http://ctan.math.washington.edu/tex-archive/graphics/pgf/contrib/tikz-3dplot/tikz-3dplot_documentation.pdf

% reals set symbols
\usepackage{amssymb}

\title{Intro to Linear Algebra: Workshop 3}

\date{May 2020}

\begin{document}


\maketitle

\section{}
\subsection{}
A sinple point.\newline
\tdplotsetmaincoords{70}{110}
\begin{tikzpicture}[tdplot_main_coords, scale = 4]
\draw[thick, ->] (0,0,0) -- (1,0,0) node[anchor=north east]{$x$};
\draw[thick, ->] (0,0,0) -- (0,1,0) node[anchor=north west]{$y$};
\draw[thick, ->] (0,0,0) -- (0,0,1) node[anchor=south]{$z$};
\fill[red] ($(0,0,0)$) circle[radius=0.5pt] node[anchor=north west]{$(0,0,0)$};
\end{tikzpicture}

\subsection{}
A line.\newline
\tdplotsetmaincoords{70}{110}
\begin{tikzpicture}[tdplot_main_coords, scale = 2]
\draw[thick, ->] (0,0,0) -- (2,0,0) node[anchor=north east]{$x$};
\draw[thick, ->] (0,0,0) -- (0,2,0) node[anchor=north west]{$y$};
\draw[thick, ->] (0,0,0) -- (0,0,2) node[anchor=south]{$z$};
\draw[blue, thick, ->] (0,0,0) -- (1,1,0) node[anchor=south]{$(1,1,0)$};

\draw[color=red] (-2,-2,0) -- (2,2,0);
\end{tikzpicture}

\subsection{}
A line.\newline
\tdplotsetmaincoords{70}{110}
\begin{tikzpicture}[tdplot_main_coords, scale = 2]
\draw[thick, ->] (0,0,0) -- (2,0,0) node[anchor=north east]{$x$};
\draw[thick, ->] (0,0,0) -- (0,2,0) node[anchor=north west]{$y$};
\draw[thick, ->] (0,0,0) -- (0,0,2) node[anchor=south]{$z$};
\draw[blue, thick, ->] (0,0,0) -- (1,0,0) node[anchor=west]{$(1,0,0)$};
\draw[blue, thick, ->] (0,0,0) -- (-1,0,0) node[anchor=west]{$(-1,0,0)$};

\draw[color=red] (-2,0,0) -- (3,0,0);
\end{tikzpicture}

\subsection{}
A plane.\newline
\tdplotsetmaincoords{70}{140}
\begin{tikzpicture}[tdplot_main_coords, scale = 2]
\draw[thick, ->] (0,0,0) -- (2,0,0) node[anchor=north east]{$x$};
\draw[thick, ->] (0,0,0) -- (0,2,0) node[anchor=north west]{$y$};
\draw[thick, ->] (0,0,0) -- (0,0,2) node[anchor=south]{$z$};
\draw[blue, thick, ->] (0,0,0) -- (1,0,0) node[anchor=south]{$(1,0,0)$};
\draw[blue, thick, ->] (0,0,0) -- (0,0,1) node[anchor=west]{$(0,0,1)$};
\fill[red,opacity=0.3]
  (2,0,2) --(-2,0,2)--(-2,0,-2)--(2,0,-2) -- cycle;  
\end{tikzpicture}


\subsection{}
A plane.\newline
\tdplotsetmaincoords{70}{140}
\begin{tikzpicture}[tdplot_main_coords, scale = 2]
\draw[thick, ->] (0,0,0) -- (2,0,0) node[anchor=north east]{$x$};
\draw[thick, ->] (0,0,0) -- (0,2,0) node[anchor=north west]{$y$};
\draw[thick, ->] (0,0,0) -- (0,0,2) node[anchor=south]{$z$};
\draw[blue, thick, ->] (0,0,0) -- (1,0,0) node[anchor=south]{$(1,0,0)$};
\draw[blue, thick, ->] (0,0,0) -- (1,0,1) node[anchor=west]{$(1,0,1)$};

\fill[red,opacity=0.3]
  (2,0,2) --(-2,0,2)--(-2,0,-2)--(2,0,-2) -- cycle;  
\end{tikzpicture}

\subsection{}
A plane.\newline
\tdplotsetmaincoords{60}{160}
\begin{tikzpicture}[tdplot_main_coords, scale = 2]
\draw[thick, ->] (0,0,0) -- (2,0,0) node[anchor=north east]{$x$};
\draw[thick, ->] (0,0,0) -- (0,2,0) node[anchor=north west]{$y$};
\draw[thick, ->] (0,0,0) -- (0,0,2) node[anchor=south]{$z$};
\draw[blue, thick, ->] (0,0,0) -- (1,1,0) node[anchor=south]{$(1,1,0)$};
\draw[blue, thick, ->] (0,0,0) -- (-1,-1,-1) node[anchor=south]{$(-1,-1,-1)$};
\draw[blue, thick, ->] (0,0,0) -- (0,0,1) node[anchor=west]{$(0,0,1)$};

\fill[red,opacity=0.3]
  (2,2,2) -- (2,2,-2) -- (-2,-2,-2) -- (-2,-2,2) -- cycle;  
\end{tikzpicture}

\subsection{} 
Entire 3D space.\newline
\tdplotsetmaincoords{60}{130}
\begin{tikzpicture}[tdplot_main_coords, scale = 2]
\draw[thick, ->] (0,0,0) -- (2,0,0) node[anchor=north east]{$x$};
\draw[thick, ->] (0,0,0) -- (0,2,0) node[anchor=north west]{$y$};
\draw[thick, ->] (0,0,0) -- (0,0,2) node[anchor=south]{$z$};
\draw[blue, thick, ->] (0,0,0) -- (1,1,0) node[anchor=south]{$(1,1,0)$};
\draw[blue, thick, ->] (0,0,0) -- (1,0,1) node[anchor=south]{$(1,0,1)$};
\draw[blue, thick, ->] (0,0,0) -- (0,0,1) node[anchor=west]{$(0,0,1)$};

\fill[red,opacity=0.3]
  (2,-2,2) -- (2,-2,-2) -- (-2,2,-2) -- (-2,2,2) -- cycle;  
\end{tikzpicture}

\section{}
\subsection{}
A single 0.

\subsection{}
$n + nx$ for integer $n$.

\subsection{}
All integers.

\subsection{}
$n + mx^2$ for integers $n, m$.

\subsection{}
$n + mx^2$ for integers $n, m$.

\subsection{}
$n + nx + mx^2$ for integers $n, m$.

\subsection{}
$n + mx + tx^2$ for integers $n, m, t$.

\section{}
\subsection{}
a, b, d, e and g are linearly independent. c and f are dependent.
\subsection{}
Every linear independent set in $\mathbb{R}$ spans into same dimension of the cardinality of the set except of {(0,0,0)} (which spans to a point). Such set with cardinality 1 spans into a line, 2 into a plane and 3 the entire 3D space. To know what a set in $\mathbb{R}$ spans, we can remove dependent vectors from the set first.
\subsection{}
a, b, d, e and g are linearly independent. c and f are dependent.

\section{}
Yes, in fact we could form a bijection from $\mathcal{P}_m$ to $\mathbb{R}^{m+1}$ by doing the following. For each non-negative$n$:
$$a_0 + a_1 x + ... + a_n x^n \Longleftrightarrow (a_0, a_1, ... , a_n)$$\newline
If we apply this to problem two we would get the sets:\newline
a.$\{(0,0,0)\}$\newline
b.$\{(1,1,0)\}$\newline
c.$\{(1,0,0),(−1,0,0)\}$\newline
d.$\{(1,0,0),(0,0,1)\}$\newline
e.$\{(1,0,0),(1,0,1)\}$\newline
f.$\{(1,1,0),(0,0,1),(−1,−1,−1)\}$\newline
g.$\{(1,1,0),(0,0,1),(1,0,1)\}$\newline
Which is exactly as problem 1.
\end{document}